\chapter{Johdanto}%
\label{ch:johdanto}
Dieselmoottori on muun muassa toimintavarmuutensa ja energiatehokkuutensa vuoksi yksi nykypäivän merkittävimmistä voimanlähteistä.  
\cite[s. 121, 137-138]{Koten_2024}.
Pelkästään Yhdysvalloissa noin 75\% maatalouden työkoneista käyttää dieselmoottoria voimanlähteenään \cite[s. 122]{Koten_2024}.  
Dieselmoottori kuitenkin tuottaa palamisolosuhteiden vuoksi
huomattavasti päästöjä \cite{FiebigMichael2014Pefd}. Erityisesti typen oksidi- (NO\(_x\)) ja hiukkaspäästöt (PM) ovat ongelma niin terveydelle kuin ympäristöllekin \cite{YaoDongwei2023Rodm}\cite[s. 138]{Koten_2024}, joten  lainsäädäntö asettaa yhä tiukentuvat rajat dieselmoottorien päästöille. Tämän vuoksi moderneissa työkoneissa on käytettävä pakokaasun jälkikäsittelyjärjestelmiä, jotka pyrkivät minimoimaan dieselmoottorin tuottamien haitallisten päästöjen pääsyä ilmakehään. Koska lainsäädännön rajat tiukentuvat, täytyy jälkikäsittelyjärjestelmiä kehittää ja parantaa jatkuvasti.

Hiukkaspäästöjä voidaan pienentää merkittävästi hiukkassuodattimella (DPF, \emph{eng. Diesel Particulate Filter}).
Tyypillisin \cite{SHIYunxi2020Eota} DPF-järjestelmän rakenne on hunajakennomainen sylinteri, jossa on vuoronperään eri päistä suljettuja putkia. Putkien välillä on huokoiset seinämät, joiden läpi pakokaasu pääsee virtaamaan.
 %DPF suodattaa läpivirtaavasta pakokaasusta jopa 99\% hiukkaslukumäärästä ja 95\% -massasta \cite{Yan_state_of_the_art}. 
DPF suodattaa läpivirtaavasta pakokaasusta hiilestä ja hiilivedyistä koostuvan noen sekä palamattomasta materiaalista koostuvan tuhkan huokoisiin seinämiin ja niiden pinnalle. 

DPF-järjestelmään kertynyt noki ja tuhka tulee aika ajoin poistaa. Tuhka poistetaan manuaalisella puhdistuksella, mutta nokea voidaan hapettaa hiilidioksidiksi hapen ja typen oksidien avulla. Tätä kutsutaan regeneroinniksi, joka jaetaan karkeasti passiiviseen ja aktiiviseen regenerointiin.
Aktiivisen regeneroinnin ajoitus on syytä toteuttaa oikeaan aikaan, sillä liian aikainen regenerointi kuluttaa ylimäärin polttoainetta, kun taas liian myöhäinen nostaa vastapainetta, mikä voi olla jopa vaarallista \cite{YaoDongwei2023Rodm}. 


Näin ollen DPF-järjestelmän tila -- erityisesti noen määrä -- on syytä tuntea mahdollisimman tarkasti. 

Noen määrää arvioidaan 

