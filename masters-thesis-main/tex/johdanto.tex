\chapter{Johdanto}%
\label{ch:johdanto}
Dieselmoottori on muun muassa toimintavarmuutensa ja energiatehokkuutensa vuoksi yksi nykypäivän merkittävimmistä voimanlähteistä muun muassa työkoneissa \cite[s. 121, 137-138]{Koten_2024}.
Pelkästään Yhdysvalloissa noin 75\% maatalouden työkoneista käyttää dieselmoottoria voimanlähteenään \cite[s. 122]{Koten_2024}.  
Dieselmoottori kuitenkin tuottaa palamisolosuhteiden vuoksi
huomattavasti päästöjä \cite{FiebigMichael2014Pefd}. Erityisesti typen oksidit (NO\(_x\)) ja hiukkaspäästöt ovat ongelma niin terveydelle kuin ympäristöllekin \cite[s. 138]{Koten_2024}\cite{YaoDongwei2023Rodm}, joten  lainsäädäntö asettaa yhä tiukentuvat rajat dieselmoottorien päästöille. Tämän vuoksi moderneissa työkoneissa on käytettävä pakokaasun jälkikäsittelyjärjestelmiä, jotka pyrkivät vähentämään dieselmoottorin tuottamien haitallisten päästöjen pääsyä ilmakehään. Koska lainsäädännön rajat tiukentuvat, täytyy jälkikäsittelyjärjestelmiä kehittää ja parantaa jatkuvasti.

Hiukkaspäästöjä voidaan pienentää merkittävästi hiukkassuodattimella (DPF, \emph{eng. Diesel Particulate Filter}).
Tyypillisin DPF-järjestelmän rakenne on hunajakennomainen sylinteri, jossa on vuoronperään eri päistä suljettuja putkia \cite{SHIYunxi2020Eota}. Putkien välillä on huokoiset seinämät, joiden läpi pakokaasu pääsee virtaamaan. {\color{red} TODO}

DPF suodattaa läpivirtaavasta pakokaasusta hiilestä ja hiilivedyistä koostuvan noen sekä palamattomasta materiaalista koostuvan tuhkan huokoisiin seinämiin ja niiden pinnalle. {\color{red} TODO}

DPF-järjestelmään kertynyt noki ja tuhka tulee aika ajoin poistaa. Tuhka poistetaan manuaalisella puhdistuksella, mutta nokea voidaan hapettaa hiilidioksidiksi hapen ja typen oksidien avulla. Tätä kutsutaan regeneroinniksi, joka jaetaan karkeasti passiiviseen ja aktiiviseen regenerointiin.
Aktiivisen regeneroinnin ajoitus on syytä toteuttaa oikeaan aikaan, sillä liian aikainen regenerointi kuluttaa ylimäärin polttoainetta, kun taas liian myöhäinen nostaa vastapainetta, mikä voi olla jopa vaarallista \cite{YaoDongwei2023Rodm}. 
Näin ollen DPF-järjestelmän tila -- erityisesti noen  määrä -- on syytä tuntea mahdollisimman tarkasti. 

Nokilatausta arvioidaan painehäviömittauksilla ja hiukkaslukumäärää kuvaavaan matemaattisfysikaaliseen malliin perustuvalla ennusteella \cite{YaoDongwei2023Rodm}. Painehäviömittaus on kuitenkin herkkä häiriöille ja vikaantumiselle, eikä se kykene erottelemaan nokea ja tuhkaa toisistaan.
Toisaalta mallin ennusteen virhe kumuloituu ajan myötä \cite{YaoDongwei2023Rodm}.
Tämän työn tarkoituksena on toteuttaa näitä arvioita yhdistelemällä hyvä nokilatausestimaatti. 

Hiukkassuodattimeen kertyneen tuhkan määrää arvioidaan mallipojaisesti \cite{??}. 
Kertynyt tuhka aiheuttaa painemittaukseen kasvavan trendin. Työssä on tarkoitus toteuttaa estimointimenetelmä tuhkalataukselle painemittaukseen perustuen.{\color{red} TODO}

Estimointiin käytetään Bayesilaisia suotimia, erityisesti Kalman-suodinta. Hiukkassuodatinta mallintavien yhtälöiden epälineaarisuuden vuoksi työssä tarkastellaan Kalman-suo-timen epälineaarisia versioita, kuten laajennettua Kalman-suodinta (EKF, \emph{Extended Kalman Filter}).{\color{red} TODO}

Työssä ei suoriteta uusia mittauksia, vaan käytetään usean kenttäkäyttöisen koneen moottorinohjausyksikön (ECU, \emph{eng. Engine Control Unit}) keräämää dataa. {\color{red} TODO}

Työn tavoitteena on selvittää, kuinka paljon hapettavan katalysaattorin \(\ce{NO}_x\)-mallin virhe aiheuttaa epävarmuutta DPF-tilan estimaattiin. Epävarmuutta mallinnetaan tilan todennäköisyysjakauman varianssina. Toisaalta kiinnostavaa on myös selvittää, kuinka paljon hiukkassuodattimen käyttöolosuhteet, kuten lämpötila vaikuttavat DPF-tilan varianssiin. 

Luvussa 2 esitellään hiukkassuodatin yksityiskohtaisesti osana jälkikäsittelyjärjestelmää.
Luvussa 3 esitellään yleisesti systeemin tilan estimointimenetelmiä, erityisesti Kalman-suodin ja sen laajennuksia.
Luvussa 4 toteutetaan luvun 3 mukainen estimointimenetelmä hiukkassuodattimen tilan estimoinniksi. Luvussa 5 tarkastellaan saatuja tuloksia.