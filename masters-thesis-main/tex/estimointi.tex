\chapter{Systeemin tilan estimointi }%Bayesilaisilla suotimilla}%
\label{ch:estimointi}
Optimaaliset suotimet ovat  matemaattisia menetelmiä, joilla estimoidaan systeemien tiloja kohinaisiin mittauksiin perustuen \cite{sarkka_bayesian}. Tässä työssä tarkastellaan Bayesilaisia suotimia, jotka ratkaisevat optimointiongelman Bayesilaisittain. 

\section{Lyapunov ja Riccati}
{\color{red} Näillä teoreettista taustaa (nimi vaihtuu)}

\section{Kalman-suodin}

Kalman-suodin on algoritmi, jolla voidaan estimoida lineaarisen ja kohinaisen systeemin tilaa. Systeemin kohinan tulee olla valkoista kohinaa \cite[s. 56]{sarkka_bayesian}. Algoritmi on kaksivaiheinen ja se koostuu ennuste- ja päivitysvaiheista. 
Ennustevaihe tehdään lähtökohtaisesti jokaisella aika-askeleella. Päivitysvaihe sen sijaan voidaan laskennan tai saavuttamattoman mittauksen vuoksi toteuttaa harvemminkin. 

Tarkastellaan lineaarista systeemiä
\begin{align}
    \begin{split}
        x_k &= A_{k-1}x_{k-1} + q_{k-1} \\
        y_k &= H_k x_k + r_k,
    \end{split}
\end{align}
jossa \(x_k \in \R^n \) on systeemin tila hetkellä \(k\), \(A_{k-1}\in \R^{n\times n}\) on tilamatriisi, \(y_k\in \R^m\) on mittaus, \(H_k \in \R^{m \times n}\) on mittausmatriisi. Muuttujat \(q_{k-1} \sim \mathcal{N}(0, Q_{k-1})\) ja \(r_k \sim \mathcal{N}(0, R_k)\) ovat normaalijakautuneet (Gaussiset) prosessi- ja mittauskohinat, joissa \(Q_{k-1}\) ja \(R_k\) ovat vastaavat kovarianssimatriisit. Tilan ja mittauksen estimaatteja merkitään \(\hat{x}_k\) ja \(\hat{y}_k\).

Kalman-suodin ratkaisee jokaisella iteraatiolla normaalijakautuneen estimaatin, jossa keskiarvo on \(\hat{x}_k\) ja optimoitu kovarianssi on \(P_k\).  

Ennustevaihe:
\begin{align}
    \hat{x}_{k | k-1}  &= A_{k-1} \hat{x}_{k-1}\\
    P_{k | k-1} &= A_{k-1} P_{k-1} A_{k-1}^T + Q_{k-1}.
\end{align}

Päivitysvaihetta varten tarvitaan mittausennuste
\begin{align}
    \hat{y}_k = H_k \hat{x}_{k | k-1} 
\end{align}
ja Kalman-vahvistus 
\begin{align}
    K_k = P_{k | k-1} H_k^T \left(H_k P_{k | k-1} H_k^T + R_k \right)^{-1}.
\end{align}
Kalman-vahvistuksen yhtälö minimoi todellisen ja estimoidun tilan keskiarvon neliöllisen virheen \cite{sparse_kalman_gain}, ja se määrittää kuinka paljon uuteen mittaukseen uskotaan vanhaan estimaattiin verrattuna \cite{becker2023kalman}.

Päivitysvaihe:
\begin{align}
    \hat{x}_{k | k} &= \hat{x}_{k | k-1} + K_k (y_k - \hat{y}_k)\\
    P_{k|k}         &= (I - K_k H_k)P_{k|k-1},
\end{align}
jossa \(I \in \R^{n\times n}\) on identiteettimatriisi.

\section{EKF}
Toisin kuin tavallinen Kalman-suodin, laajennettu Kalman-suodin, eli EKF \emph{(eng. Extended Kalman Filter)} kykenee ratkaisemaan epälineaaristen funktioiden optimointiongelman lähes optimaalisesti. Menetelmä perustuu funktioiden linearisointiin, joten ratkaisu ei välttämättä ole täysin optimaalinen, mutta usein riittävän optimaalinen. Menetelmä kuitenkin vaatii funktioiden derivoituvuutta, sillä algoritmissa ratkaistaan sekö tilanfunktion että mittausfunktion Jacobin matriisit joka iteraatiolla. Tarkastellaan diskreettiä systeemiä, jonka tilanyhtälö on
\begin{align}
    x_k = f(x_{k-1}, u_{k-1}) + w_{k-1},
\end{align}
ja mittausyhtälö on
\begin{align}
    z_k = h(x_{k}, u_{k}) + v_k.
\end{align}
Funktiot \(f\) ja \(h\) ovat yleisesti epälineaarisia ja derivoituvia. Prosessi- ja mittaushäiriöt \(w\) ja \(v\) ovat nollakeskiarvoisia ja normaalijakautuneita, ja niiden kovarianssimatriiseita merkitään \(Q\) ja \(R\).

Kuten tavallisellakin Kalman-suotimella, EKF-algoritmi voidaan jakaa ennuste- ja päivitysvaiheisiin. 
Ennustevaiheessa ratkaistaan uusi estimaatti sekä sen kovarianssimatriisi. Ennustevaihe on muotoa
\begin{align}
    \hat{x}_{k|k-1} &= f(\hat{x}_{k-1|k-1}, u_{k-1}) \\
    P_{k|k-1} &= F_k P_{k-1|k-1} F_k^T + Q_{k-1},
\end{align}
jossa \(F\) on funktion \(f\) Jacobin matriisi. 
Päivitysvaiheessa tilan päivitystä varten ratkaistaan Kalman-vahvistus
\begin{align}
    S_k &= H_k P_{k|k-1} H_k^T + R_k \\
    K_k &= P_{k|k-1} H_k^T S_k^{-1},
\end{align}
jossa \(H\) on funktion \(h\) Jacobin matriisi. Kalman-vahvistuksella ja mittauksella \(z\) voidaan päivittää estimaatti ja sen kovarianssi
\begin{align}
    \hat{x}_{k|k} &= \hat{x}_{k|k-1} + K_k (z_k - h(\hat{x}_{k|k-1})) \\
    P_{k|k} &= (I-K_k H_k) P_{k|k-1}.
\end{align}
Kalman-vahvistuksen epäoptimaalisuus johtuu linearisointien epätarkkuudesta. Täten EKF ei ole luotettava estimointialgoritmi pitkille aikaikkunoille, hitaille näytteenottotaajuuksille tai huonosti derivoituville funktioille.