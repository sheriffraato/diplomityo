
\chapter{Painehäviömallin linearisointi}%
\label{ch:dP_linearisointi}

Tässä liitteessä johdetaan painehäviömallin yhtälöiden linearisointi. Yleisesti linearisointi kohdassa \(x_e\) voidaan esittää muodossa
\begin{align}
    %\tilde{f}(x) = f(x_e) +\sum_{k=1}^n \frac{\doo f(x_e)}{\doo x_k} (x_k-x_e),
    \tilde{f}(x) = f(x_e) +J_f (x_e) (x_k-x_e),
\end{align}
jossa \(x, x_e \in \R^k\) ja \(J_f\) on funktion \(f\) Jacobin matriisi. 
Kokonaispainehäviön termeistä vain \(\Delta P_{inlet}\), \(\Delta P_{soot}\) ja \(\Delta P_{ash}\) riippuvat noki- ja tuhkakerroksista \(w_{soot}\) ja \(w_{ash}\). 
Linearisointi toteutetaan muuttujien \(m_{ash}\) ja \( m_{soot}\) suhteen, joten
sovelletaan yleistä ketjusääntöä Jacobin matriiseille
\begin{align}
    J_{f \circ g}(x) = J_f\big(g(x)\big)  J_g(x).
\end{align}
% \(    J_{f \circ g}(x) = J_f\big(g(x)\big)  J_g(x).\)
Olkoon \(m = \left(m_{soot}, m_{ash}\right) \in \R^2\) ja \(w(m) = \big(w_{soot}(m), w_{ash}(m)  \big) \in \R^2\). Laaditaan funktion \(w\) Jacobin matriisi
\begin{align}
    J_w(m)=
    \bm{
        \frac{\doo w_{soot}}{\doo m_{soot}}
        & 
        \frac{\doo w_{soot}}{\doo m_{ash}} 
        \\  
        \frac{\doo w_{ash}}{\doo m_{soot}}
        & 
        \frac{\doo w_{ash}}{\doo m_{ash}} 
    }.
\end{align}
%joka vastaa derivaattaa.
 Funktion \(\Delta P_{tot}\) Jacobin matriisi puolestaan on 
\begin{align}
    J_{\Delta P_{tot}}(w) = \bm{  
        \frac{\doo \Delta P_{tot}}{\doo w_{soot}}
        &   
        \frac{\doo \Delta P_{tot}}{\doo w_{ash}}
    }.
\end{align}
Näin ollen 
\begin{align}\label{eq:jacob_total_general}
    \begin{split}
        J_{\Delta P_{tot}}(m)&=
        J_{\Delta P}\big(w(m)\big)  J_{w}(m)
\\ &=
    \bm{  
        \frac{\doo \Delta P_{tot}}{\doo w_{soot}}
        &   
        \frac{\doo \Delta P_{tot}}{\doo w_{ash}}
        }
    \bm{
        \frac{\doo w_{soot}}{\doo m_{soot}}
        & 
        \frac{\doo w_{soot}}{\doo m_{ash}} 
        \\ 
        \frac{\doo w_{ash}}{\doo m_{soot}}
        & 
        \frac{\doo w_{ash}}{\doo m_{ash}} 
        }
\\ &=
    \bm{
        \frac{\doo \Delta P_{tot}}{\doo w_{ash}}
        \frac{\doo w_{ash}}{\doo m_{soot}}
        +
        \frac{\doo \Delta P_{tot}}{\doo w_{soot}}
        \frac{\doo w_{soot}}{\doo m_{soot}}
        &
        \frac{\doo \Delta P_{tot}}{\doo w_{ash}}
        \frac{\doo w_{ash}}{\doo m_{ash}}
        +
        \frac{\doo \Delta P_{tot}}{\doo w_{soot}}
        \frac{\doo w_{soot}}{\doo m_{ash}}
        } 
\\ &= 
    \bm{
        \frac{\doo \Delta P_{tot}}{\doo m_{soot}}
        & 
        \frac{\doo \Delta P_{tot}}{\doo m_{ash}} 
        }.
    \end{split}
\end{align}
% Linearisointi on muotoa
% \begin{align}
%     \begin{split}
%     \Delta\tilde{ P}_{tot} &= \Delta P_{tot}
%     +
%     \frac{\doo {\Delta P}_{tot}}{\doo m_{ash}} \Delta m_{ash}
%     + 
%     \frac{\doo {\Delta P}_{tot}}{\doo m_{soot}}  \Delta m_{soot},
% \end{split}
% \end{align}
% joten r
Ratkaistaan Jacobin matriisi 
derivoimalla yhtälöt 
\eqref{eq:deltaP_soot} ja \eqref{eq:deltaP_ash}
noki- ja tuhkalatauksen suhteen. 
% Tuhkakerroksen paksuuden linearisointi on 
% \begin{align}
%     \tilde{w}_{soot}( m_{ash},  m_{soot}) = \frac{\doo w_{soot}}{\doo m_{ash}}  \Delta m_{ash}
%                                                     + \frac{\doo w_{soot}}{\doo m_{soot}} \Delta m_{soot},
% \end{align} 
% ja nokikerroksen 
% \begin{align}
%     \tilde{w}_{ash}( m_{ash},  m_{soot}) = \frac{\doo w_{ash}}{\doo m_{ash}}  \Delta m_{ash}
%                                                     + \frac{\doo w_{ash}}{\doo m_{soot}} \Delta m_{soot}.
% \end{align}
% Yhtälö ei riipu noen määrästä, joten
% \begin{align}
%     \tilde{w}_{soot}( m_{ash},  m_{soot}) =  \frac{\Delta m_{soot}}
%     {4L n_{open} \rho_{soot}\sqrt{\alpha_{in}^2 - \frac{m_{soot, 0}}{L n_{open} \rho_{soot}}}}.
% \end{align}

% Nokikerroksen paksuuteen vaikuttaa myös tuhkakerroksen paksuus. Sijoittamalla yhtälö \eqref{eq:deltaP_wsoot} yhtälöön \eqref{eq:deltaP_wash}, saadaan
% \begin{align}
%     w_{ash}(m_{ash}, m_{soot}) =
%     \frac{ \sqrt{\alpha_{in}^2 - \frac{m_{soot}}{L n_{open} \rho_{soot}}}
%             - 
%            \sqrt{\alpha_{in}^2 - \frac{m_{soot}}{L n_{open} \rho_{soot}} - \frac{m_{ash}}{L n_{open} \rho_{ash}}} }
%     {2}.
% \end{align}

% \begin{align}
%     \begin{split}
%     \tilde{w}_{ash}(m_{ash}, m_{soot}) = &
%     \frac{\Delta m_{ash}}{4 L n_{open} \rho_{{ash}} \sqrt{a_{{in}}^2 - \frac{m_{{soot,0}}}{L n_{open} \rho_{{soot}}} - \frac{m_{{ash,0}}}{L n_{open} \rho_{{ash}}}}} + \\
%     & \frac{\Delta m_{soot}}{4 L n_{open} \rho_{{soot}} \sqrt{a_{{in}}^2 - \frac{m_{{soot,0}}}{L n_{open} \rho_{{soot}}} - \frac{m_{{ash,0}}}{L n_{open} \rho_{{ash}}}}} - \\ &
%     \frac{\Delta m_{soot}}{4 L n_{open} \rho_{{soot}} \sqrt{a_{{in}}^2 - \frac{m_{{soot,0}}}{L n_{open} \rho_{{soot,0}}}}}
%     \end{split}
% \end{align}

% Paremmin näin


% \begin{align*}
%     a &:= 4 L n_{open}, \\
%     b &:= \sqrt{\alpha_{in}^2 - \frac{m_{soot}}{L n_{open} \rho_{soot}}}, \\
%     c &:= \sqrt{\alpha_{in}^2 - \frac{m_{soot}}{L n_{open} \rho_{soot}} - \frac{m_{ash}}{L n_{open} \rho_{ash}}}, \\
%     d &:= \alpha_{in} + \alpha_{out} + w_s,\\
%     f &:= \alpha_{out}- 2 w_{soot},\\
%     q &:= \frac{1}{2}Q\mu,\\
%     \alpha_{in}^* &:= \alpha_{in} - 2 w_{soot} - 2 w_{ash} , \\
%     \alpha_{out}^* &:= \alpha_{out} - 2 w_{soot}  - 2 w_{ash}.
% \end{align*}

% Esitellään apumuuttujat
% \begin{center}
% \begin{tabular}{ll}
%     \( a := 4 L n_{open} \) & \( f := \alpha_{out} - 2 w_{soot} \) \\
%     \( b := \sqrt{\alpha_{in}^2 - \frac{m_{soot}}{L n_{open} \rho_{soot}}} \) & \( q := \frac{1}{2} Q \mu \) \\
%     \( c := \sqrt{\alpha_{in}^2 - \frac{m_{soot}}{L n_{open} \rho_{soot}} - \frac{m_{ash}}{L n_{open} \rho_{ash}}} \) & \( \alpha_{in}^* := \alpha_{in} - 2 w_{soot} - 2 w_{ash} \) \\
%     \( d := \alpha_{in} + \alpha_{out} + 2w_s \) & \( \alpha_{out}^* := \alpha_{out} - 2 w_{soot} - 2 w_{ash} \)
% \end{tabular}
% \end{center}


% Tuhkakerroksen derivaatat ovat
% \begin{align}
%     \frac{\doo w_{soot}}{\doo m_{ash}} & =0,\\
%     \frac{\doo w_{soot}}{\doo m_{soot}} &= \frac{1}{ab\rho_{soot}},
% \end{align}
% ja vastaavasti nokikerroksen derivaatat ovat
% \begin{align}
%     \frac{\doo w_{ash}}{\doo m_{ash}} &= \frac{1}{ac\rho_{ash}}\\
%     \begin{split}
%     \frac{\doo w_{ash}}{\doo m_{soot}} &= \frac{1}{ac\rho_{soot}}-\frac{1}{ab\rho_{soot}}
%     \\ &= \frac{ab \rho_{soot} - ac\rho_{soot}}{ab\rho_{soot}\cdot ac \rho_{soot}}
%     \\ &= \frac{\cancel{a\rho}_{soot}(b-c)}{a^{\cancel{2}} \rho_{soot}^{\cancel{2}} bc}
%     \\ &= \frac{b-c}{abc\rho_{soot}}.
%     \end{split}
% \end{align}

% Painehäviön derivaatta on
% \begin{align}
%     \nabla \Delta P_{tot} &=
%     \nabla \Delta P_{inlet} + \nabla \Delta P_{ash} + \nabla \Delta P_{soot}.
% \end{align}
% Lasketaan termien derivaatat erikseen. Sisäänmenossa derivaatat ovat
% \begin{align}
%     \frac{\doo \Delta P_{inlet}}{\doo w_{ash}} &=
%     \frac{\doo \Delta P_{inlet}}{\doo w_{soot}} =
%     \frac{8 F L^2 d^2 q}{3 V \alpha_{in}^{*5}} .
% \end{align}
% Nokikerroksen aiheuttaman painehäviön derivaatat ovat
% \begin{align}
%     \frac{\doo \Delta P_{ash}}{\doo w_{ash}} &=
%     \frac{2 q}{2 \kappa_{ash} \alpha_{out}^*},
%     \\\frac{\doo \Delta P_{ash}}{\doo w_{soot}}
%     &=
%     \frac{4qw_{ash}}{a f\kappa_{ash} \alpha_{out}^* }.
% \end{align}
% Tuhkakerros:
% \begin{align}
%     \frac{\doo \Delta P_{soot}}{\doo w_{ash}} &= 0,
%     \\\frac{\doo \Delta P_{soot}}{\doo w_{soot}} &=
%     \frac{2q}{a f \kappa_{soot}  }.
% \end{align}

% Näin ollen 
% \begin{align}
%     \frac{\doo \Delta P_{tot}}{\doo m_{ash}} &=
%      \frac{1}{ac\rho_{ash}}
%     \left(\frac{8 F L^2 d^2 q}{3 V \alpha_{in}^{*5}} +
%      \frac{2 q}{2 \kappa_{ash} \alpha_{out}^*}
%     \right)\\
% % \end{align}
% % \begin{align}
% \begin{split}
%     \frac{\doo \Delta P_{tot}}{\doo m_{soot}} &=
%     \frac{b-c}{abc\rho_{soot}}\left(\frac{8 F L^2 d^2 q}{3 V \alpha_{in}^{*5}} +
%      \frac{2 q}{2 \kappa_{ash} \alpha_{out}^*}
%     \right) \\
%     & + 
%     \frac{1}{ab\rho_{soot}} \left(
%         \frac{8 F L^2 d^2 q}{3 V \alpha_{in}^{*5}} +
%         \frac{4qw_{ash}}{a f \kappa_{ash} \alpha_{out}^*  } +
%         \frac{2q}{a f \kappa_{soot}  }
%     \right).
% \end{split}
% \end{align}
% \newpage


% Esitellään apumuuttujat
% \begin{align*}
% S(w_{{ash}}, w_{{soot}}) &=\alpha_{{out}}-2\,w_{{soot}}-2\,w_{{ash}},\\
% T(w_{{ash}}, w_{{soot}}) &=\alpha_{{in}}-2\,w_{{soot}}-2\,w_{{ash}},\\
% U(w_{soot}) &=  \alpha_{out} - 2w_{soot}, \\
% X(m_{{soot}})            &=\alpha_{{in}}^2 -\frac{m_{{soot}}}{L\,n_{{open}}\,\rho_{{soot}}},\\
% Y(m_{{ash}}, m_{{soot}}) &=X-\frac{m_{{ash}}}{L\,n_{{open}}\,\rho_{{ash}}},\\ 
% M &=4\,L\,n_{{open}},\\
% P &=\alpha_{{in}}+\alpha_{{out}}+2\,w_s,\\
% A &=\frac{Q\,\mu}{M\,k_{{ash}}\,S},\\
% B &= \frac{4\,F\,L^2\,Q\,\mu\,P^2}{3\,V\,T^5},\\
% C &= A + B,\\
% D &=M\,\rho_{{ash}}\,\sqrt{Y},\\
% E &=\frac{1}{M\,\rho_{{soot}}\,\sqrt{X}},\\
% F_v &= \frac{1}{M\,\rho_{{soot}}\,\sqrt{Y}},\\
% G &=\frac{Q\,\mu}{M\,k_{soot}\,U},\\
% H &=\frac{Q\,\mu}{M\,k_{ash}\,U}
%       \Bigl(\frac{\,U}{S}+1\Bigr).
% \end{align*}

% Derivaattoja
% \begin{align}
%     \frac{\doo w_{ash}}{\doo m_{ash}} &= \frac{1}{M \rho_{ash}\sqrt{Y}} \\
%     \frac{\doo w_{ash}}{\doo m_{soot}} &= \frac{1}{M\rho_{soot}\sqrt{Y}} - \frac{1}{M\rho_{ash}\sqrt{X}} \\
%     \frac{\doo w_{soot}}{\doo m_{ash}}&=0\\
%     \frac{\doo w_{soot}}{\doo m_{soot}} &= \frac{1}{M \rho_{soot}\sqrt{X}}.
% \end{align}

% \begin{align}
%     \frac{\doo \Delta P_{tot}}{\doo w_{ash}} &= A+B\\
%     \frac{\doo \Delta P_{tot}}{\doo w_{soot}} &= G+B-H.
% \end{align}


% \begin{align}
%         \frac{\doo \Delta P_{tot}}{\doo m_{ash}} &=\frac{C}{D},
%         \\
%         \frac{\doo \Delta P_{tot}}{\doo m_{soot}} &=
%         C\bigl(E - F_v\bigr)-E\bigl(B - G + H\bigr).
% \end{align}

% Näin ollen 
% \begin{align}
%     J_{\Delta P_{tot}} = \bm{\frac{C}{D} & \quad C\bigl(E - F_v\bigr)-E\bigl(B - G + H\bigr)}
% \end{align}

% \newpage


Tarkastellaan funktioita 
\begin{align*}
    w_{soot}(m)
        &= \frac{\alpha_{in}-2w_{ash}(m) - \sqrt{(\alpha_{in}-2w_{ash}(m))^2 - \frac{m_{soot}(t)}{ n_{open} L \rho_{soot}}}}{2}.
    \\ 
    w_{ash}(m)
        &=\frac{\alpha_{in} - \sqrt{\alpha_{in}^2 - \frac{m_{ash}(t)}{ n_{open} L \rho_{ash}}}}{2} ,
\end{align*}
Jacobin matriisin termit, eli osittaisderivaatat ovat
% jossa 
% \begin{align}
%     X(m) &= \alpha_{in}^2 -\frac{m_{ash}}{L n_{open} \rho_{ash}},
% \end{align}
\begin{align}
    \frac{\doo w_{soot}(m)}{\doo m_{soot}} &= \frac{1}{Z\rho_{soot} \sqrt{Y(m)}},
    \\
    \frac{\doo w_{soot}(m)}{\doo m_{ash}} &=
    \frac{1}{Z \rho_{ash}} \left( \frac{1}{\sqrt{Y(m)}}-\frac{1}{\sqrt{X(m)}} \right),
\end{align}
ja
\begin{align}
    \frac{\doo w_{ash}(m)}{\doo m_{soot}}&=0,
    \\
    \frac{\doo w_{ash}(m)}{\doo m_{ash}}&=\frac{1}{Z \rho_{ash} \sqrt{X(m)}},
\end{align}
jossa 
\begin{align}
    X(m) &= \alpha_{in}^2 -\frac{m_{ash}}{L n_{open} \rho_{ash}},
    \\
    \begin{split}
        Y(m) &= X(m) - \frac{m_{soot}}{L n_{open} \rho_{soot}}  
        \\&= \alpha_{in}^2 -\frac{m_{ash}}{L n_{open} \rho_{ash}}- \frac{m_{soot}}{L n_{open} \rho_{soot}}
    \end{split}
    \\
    Z &= 4 L n_{open}.
\end{align}
Jacobin matriisi on muotoa
\begin{align}
    J_{w}(m) &=
    \frac{1}{Z} 
    \bm{
        \frac{1}{\rho_{soot}\sqrt{Y(m)}}
        &
        \frac{1}{\rho_{ash} \sqrt{Y(m)}}-\frac{1}{\rho_{ash} \sqrt{X(m)}} 
        \\
        0 
        &
        \frac{1}{\rho_{ash} \sqrt{X(m)}} 
    }.
\end{align}
Tarkastellaan painehäviöyhtälön tuhka- ja nokikerrosriippuvaisia termejä
\begin{align}
    \Delta P_{inlet}(w) &=   
    \frac{(\alpha_{in}+\alpha_{out}+2 w_s)^2FL^2}{6V(\alpha_{in} -2w_{soot}(m)-2w_{ash}(m))^4 }Q(t) \mu(t)
        \\
    \Delta P_{ash}(w) &=
    \frac{Q(t)\mu(t)}{8 n_{open} L \kappa_{ash}}\ln\left(\frac{\alpha_{out}}{\alpha_{out}-2w_{ash}(m)}\right)
        \\
    \Delta P_{soot}(w) &= \frac{Q(t)\mu(t)}{8 n_{open} L \kappa_{soot}}\ln\left(\frac{\alpha_{out}-2w_{ash}(m)}{\alpha_{out}-2w_{ash}(m)-2w_{soot}(m)}\right).
\end{align}

Näiden osittaisderivaatat ovat
\begin{align}
    \frac{\doo \Delta P_{inlet}(w)}{\doo w_{soot}}
    &= 
    \frac{\doo \Delta P_{inlet}(w)}{\doo w_{ash}}
    = 
    \frac{4 F L^2  \alpha_{cell}^2 Q(t) \mu(t) }{3 V\bigl(\alpha_{in}^*(m)\bigr)^5},
\end{align}
jossa \(\alpha_{cell} = \alpha_{in} + \alpha_{out} + 2w_s\), ja \(\alpha_{in}^*(m)=\alpha_{in}- 2 w_{ash}(m)-2w_{soot}(m)\).
Tuhkakerroksen aiheuttaman painehäviön osittaisderivaatat ovat 
\begin{align}
    \frac{\doo \Delta P_{ash}}{\doo w_{soot}} &= 0 ,
    \\
    \frac{\doo \Delta P_{ash}}{\doo w_{ash}} &= 
    \frac{Q(t)\mu(t)}{Z \kappa_{ash} \alpha_{out}^{+}(m)}.
\end{align}
ja nokikerroksen
\begin{align}
    \frac{\doo \Delta P_{soot}}{\doo w_{soot}} &= 
    \frac{Q(t)\mu(t)}{Z\kappa_{soot}\alpha_{out}^*(m)},
    \\
    \frac{\doo \Delta P_{soot}}{\doo w_{ash}} &=
    \frac{2Q(t)\mu(t)w_{soot}(m)}
         {Z\kappa_{soot}\alpha_{out}^{+}(m)\alpha_{out}^*(m)},
\end{align}
jossa \(\alpha_{out}^*(m)=\alpha_{out}-2w_{ash}(m)-2w_{soot}(m)\) ja \(\alpha_{out}^{+}(m)=\alpha_{out}-2w_{ash}(m)\).
Näin ollen kokonaispainehäviön osittaisderivaatat ovat
\begin{align}
\begin{split}
    \frac{\doo \Delta P_{tot}}{\doo w_{soot}} &=
    \frac{\doo (\Delta P_{inlet}(w)+ \Delta P_{soot}(w)+ \Delta P_{ash}(w))}{\doo w_{soot}}
    \\ &=
    \frac{4 F L^2  \alpha_{cell}^2 Q(t) \mu(t) }{3V \bigl(\alpha_{in}^*(m)\bigr)^5}
    +
    0
    +
    \frac{Q(t)\mu(t)}{Z\kappa_{soot}\alpha_{out}^*(m)}
    \\ &=
    \frac{Q(t)\mu(t)}{Z}
    \left( 
    \frac{4 ZF L^2  \alpha_{cell}^2}{3 V\bigl(\alpha_{in}^*(m)\bigr)^5}
    +
    \frac{1}{\kappa_{soot}\alpha_{out}^*(m)}
    \right),
\end{split}
\\
\begin{split}
    \frac{\doo \Delta P_{tot}}{\doo w_{ash}} &=
    \frac{\doo (\Delta P_{inlet}(w)+ \Delta P_{soot}(w)+ \Delta P_{ash}(w))}{\doo w_{ash}}
    \\ &=
    \frac{Q(t)\mu(t)}{Z}
    \left(
    \frac{4 Z F L^2 \alpha_{cell}^2}{3 V\bigl(\alpha_{in}^*(m)\bigr)^5}
    +
    \frac{1}{\kappa_{ash} \alpha_{out}^{+}(m)}
    +
    \frac{2 w_{soot}(m)}{\alpha_{out}^{+}(m)\alpha_{out}^*(m)}
    \right),
\end{split}
\end{align}
ja Jacobin matriisi \eqref{eq:jacob_total_general} saadaan muotoon
\begin{align}
\begin{split}
    J_{\Delta P_{tot}}(m) =
    \frac{Q(t)\mu(t)}{Z^2} 
    &
    \bm{
        \frac{4 ZF L^2  \alpha_{cell}^2}{3 V\bigl(\alpha_{in}^*(m)\bigr)^5}
        +
        \frac{1}{\kappa_{soot}\alpha_{out}^*(m)}
        \\
        \frac{4 Z F L^2 \alpha_{cell}^2}{3 V\bigl(\alpha_{in}^*(m)\bigr)^5}
        +
        \frac{1}{\kappa_{ash} \alpha_{out}^{+}(m)}
        +
        \frac{2 w_{soot}(m)}{\alpha_{out}^{+}(m)\alpha_{out}^*(m)}
        }^T
    \\ & \cdot
    \bm{
        \frac{1}{\rho_{soot}\sqrt{Y(m)}}
        &
        \frac{1}{\rho_{ash} \sqrt{Y(m)}}-\frac{1}{\rho_{ash} \sqrt{X(m)}} 
        \\
        0 
        &
        \frac{1}{\rho_{ash} \sqrt{X(m)}} 
        }
\end{split}
\end{align}