
\chapter{Painehäviömallin linearisointi}%
\label{ch:linearisointi}

Tässä liitteessä johdetaan painehäviömallin yhtälöiden linearisointi. Yleisesti linearisointi kohdassa \(x_e\) voidaan esittää muodossa
\begin{align}
    \tilde{f}(x) = f(x_e) +\sum_{k=1}^n \frac{\doo f(x_e)}{\doo x_k} (x_k-x_e),
\end{align}
jossa \(x, x_0 \in \R^k\). 
Painehäviöt \(\Delta P_{inlet}\), \(\Delta P_{ash}\) ja \(\Delta P_{soot}\) riippuvat noki- ja tuhkakerroksista \(w_{soot}\) ja \(w_{ash}\). 
Linearisointi toteutetaan muuttujien \(m_{soot}\) ja \( m_{ash}\) suhteen, joten
sovelletaan derivoinnin ketjusääntöä
\begin{align}
    \nabla (f \circ g)(x) = \nabla f(g(x))\nabla g(x).
\end{align}
Olkoon \(m = \left(m_{soot}, m_{ash}\right) \in \R^2\) ja \(w(m) = \big(w_{soot}(m), w_{ash}(m)  \big) \in \R^2\). Laaditaan funktion \(w\) Jacobin matriisi
\begin{align}
    J_w(m)=
    \bm{
        \frac{\doo w_{soot}}{\doo m_{soot}} & \frac{\doo w_{soot}}{\doo m_{ash}}
    \\  \frac{\doo w_{ash}}{\doo m_{soot}} & \frac{\doo w_{ash}}{\doo m_{ash}}
    },
\end{align}
joka vastaa derivaattaa. Funktion \(\Delta P_{tot}(w)\) Jacobin matriisi puolestaan on 
\begin{align}
    J_{\Delta P}(w) = \bm{  \frac{\doo \Delta P_{tot}}{\doo w_{soot}}
    &   \frac{\doo \Delta P_{tot}}{\doo w_{ash}}
    }.
\end{align}
Näin ollen 
\begin{align}
    \nabla \Delta P_{tot}(m)=
    J_{\Delta P}(w)  J_{w}(m)
    =
    \bm{  \frac{\doo \Delta P_{tot}}{\doo w_{soot}}
        &   \frac{\doo \Delta P_{tot}}{\doo w_{ash}}
        }
    \bm{
        \frac{\doo w_{soot}}{\doo m_{soot}} & \frac{\doo w_{soot}}{\doo m_{ash}}
    \\  \frac{\doo w_{ash}}{\doo m_{soot}} & \frac{\doo w_{ash}}{\doo m_{ash}}
    }.
\end{align}
Linearisointiin tarvittavat derivaatat ovat 
\begin{align}
    \frac{\doo \Delta P_{tot}}{\doo m_{soot}} &=
        \frac{\doo w_{soot}}{\doo m_{soot}}
        \frac{\doo \Delta P_{tot}}{\doo w_{soot}}
        +
        \frac{\doo w_{ash}}{\doo m_{soot}}
        \frac{\doo \Delta P_{tot}}{\doo w_{ash}},
    \\
    \frac{\doo \Delta P_{tot}}{\doo m_{ash}} &=
        \frac{\doo w_{soot}}{\doo m_{ash}}
        \frac{\doo \Delta P_{tot}}{\doo w_{soot}}
        +
        \frac{\doo w_{ash}}{\doo m_{ash}}
        \frac{\doo \Delta P_{tot}}{\doo w_{ash}}.
\end{align}

Linearisointi on muotoa
\begin{align}
    \begin{split}
    \Delta\tilde{ P}_{tot} &= \Delta P_{tot}
    +
    \frac{\doo {\Delta P}_{tot}}{\doo m_{soot}} \Delta m_{soot}
    + 
    \frac{\doo {\Delta P}_{tot}}{\doo m_{ash}}  \Delta m_{ash}.
\end{split}
\end{align}


% Tuhkakerroksen paksuuden linearisointi on 
% \begin{align}
%     \tilde{w}_{ash}( m_{soot},  m_{ash}) = \frac{\doo w_{ash}}{\doo m_{soot}}  \Delta m_{soot}
%                                                     + \frac{\doo w_{ash}}{\doo m_{ash}} \Delta m_{ash},
% \end{align} 
% ja nokikerroksen 
% \begin{align}
%     \tilde{w}_{soot}( m_{soot},  m_{ash}) = \frac{\doo w_{soot}}{\doo m_{soot}}  \Delta m_{soot}
%                                                     + \frac{\doo w_{soot}}{\doo m_{ash}} \Delta m_{ash}.
% \end{align}
% Yhtälö ei riipu noen määrästä, joten
% \begin{align}
%     \tilde{w}_{ash}( m_{soot},  m_{ash}) =  \frac{\Delta m_{ash}}
%     {4L n_{open} \rho_{ash}\sqrt{\alpha_{in}^2 - \frac{m_{ash, 0}}{L n_{open} \rho_{ash}}}}.
% \end{align}

% Nokikerroksen paksuuteen vaikuttaa myös tuhkakerroksen paksuus. Sijoittamalla yhtälö \eqref{eq:deltaP_wash} yhtälöön \eqref{eq:deltaP_wsoot}, saadaan
% \begin{align}
%     w_{soot}(m_{soot}, m_{ash}) =
%     \frac{ \sqrt{\alpha_{in}^2 - \frac{m_{ash}}{L n_{open} \rho_{ash}}}
%             - 
%            \sqrt{\alpha_{in}^2 - \frac{m_{ash}}{L n_{open} \rho_{ash}} - \frac{m_{soot}}{L n_{open} \rho_{soot}}} }
%     {2}.
% \end{align}

% \begin{align}
%     \begin{split}
%     \tilde{w}_{soot}(m_{soot}, m_{ash}) = &
%     \frac{\Delta m_{soot}}{4 L n_{open} \rho_{{soot}} \sqrt{a_{{in}}^2 - \frac{m_{{ash,0}}}{L n_{open} \rho_{{ash}}} - \frac{m_{{soot,0}}}{L n_{open} \rho_{{soot}}}}} + \\
%     & \frac{\Delta m_{ash}}{4 L n_{open} \rho_{{ash}} \sqrt{a_{{in}}^2 - \frac{m_{{ash,0}}}{L n_{open} \rho_{{ash}}} - \frac{m_{{soot,0}}}{L n_{open} \rho_{{soot}}}}} - \\ &
%     \frac{\Delta m_{ash}}{4 L n_{open} \rho_{{ash}} \sqrt{a_{{in}}^2 - \frac{m_{{ash,0}}}{L n_{open} \rho_{{ash,0}}}}}
%     \end{split}
% \end{align}

% Paremmin näin
Ratkaistaan noki- ja tuhkakerrosten derivaatat noen ja tuhkan suhteen derivoimalla yhtälöt \eqref{eq:deltaP_ash} ja \eqref{eq:deltaP_soot}.
Esitellään apumuuttujat
% \begin{align*}
%     a &:= 4 L n_{open}, \\
%     b &:= \sqrt{\alpha_{in}^2 - \frac{m_{ash}}{L n_{open} \rho_{ash}}}, \\
%     c &:= \sqrt{\alpha_{in}^2 - \frac{m_{ash}}{L n_{open} \rho_{ash}} - \frac{m_{soot}}{L n_{open} \rho_{soot}}}, \\
%     d &:= \alpha_{in} + \alpha_{out} + w_s,\\
%     f &:= \alpha_{out}- 2 w_{ash},\\
%     q &:= \frac{1}{2}Q\mu,\\
%     \alpha_{in}^* &:= \alpha_{in} - 2 w_{ash} - 2 w_{soot} , \\
%     \alpha_{out}^* &:= \alpha_{out} - 2 w_{ash}  - 2 w_{soot}.
% \end{align*}
\begin{center}
\begin{tabular}{ll}
    \( a := 4 L n_{open} \) & \( f := \alpha_{out} - 2 w_{ash} \) \\
    \( b := \sqrt{\alpha_{in}^2 - \frac{m_{ash}}{L n_{open} \rho_{ash}}} \) & \( q := \frac{1}{2} Q \mu \) \\
    \( c := \sqrt{\alpha_{in}^2 - \frac{m_{ash}}{L n_{open} \rho_{ash}} - \frac{m_{soot}}{L n_{open} \rho_{soot}}} \) & \( \alpha_{in}^* := \alpha_{in} - 2 w_{ash} - 2 w_{soot} \) \\
    \( d := \alpha_{in} + \alpha_{out} + w_s \) & \( \alpha_{out}^* := \alpha_{out} - 2 w_{ash} - 2 w_{soot} \)
\end{tabular}
\end{center}


Tuhkakerroksen derivaatat ovat
\begin{align}
    \frac{\doo w_{ash}}{\doo m_{soot}} & =0,\\
    \frac{\doo w_{ash}}{\doo m_{ash}} &= \frac{1}{ab\rho_{ash}},
\end{align}
ja vastaavasti nokikerroksen derivaatat ovat
\begin{align}
    \frac{\doo w_{soot}}{\doo m_{soot}} &= \frac{1}{ac\rho_{soot}}\\
    \begin{split}
    \frac{\doo w_{soot}}{\doo m_{ash}} &= \frac{1}{ac\rho_{ash}}-\frac{1}{ab\rho_{ash}}
    \\ &= \frac{ab \rho_{ash} - ac\rho_{ash}}{ab\rho_{ash}\cdot ac \rho_{ash}}
    \\ &= \frac{\cancel{a\rho}_{ash}(b-c)}{a^{\cancel{2}} \rho_{ash}^{\cancel{2}} bc}
    \\ &= \frac{b-c}{abc\rho_{ash}}.
    \end{split}
\end{align}

Painehäviön derivaatta on
\begin{align}
    \nabla \Delta P_{tot} &=
    \nabla \Delta P_{inlet} + \nabla \Delta P_{soot} + \nabla \Delta P_{ash}.
\end{align}
Lasketaan termien derivaatat erikseen. Sisäänmenossa derivaatat ovat
\begin{align}
    \frac{\doo \Delta P_{inlet}}{\doo w_{soot}} &=
    \frac{\doo \Delta P_{inlet}}{\doo w_{ash}} =
    \frac{10 F L^2 d^2 q}{3 V \alpha_{in}^*} .
\end{align}
Nokikerroksen aiheuttaman painehäviön derivaatat ovat
\begin{align}
    \frac{\doo \Delta P_{soot}}{\doo w_{soot}} &=
    \frac{2 q}{2 \kappa_{soot} \alpha_{out}^*},
    \\\frac{\doo \Delta P_{soot}}{\doo w_{ash}}
    &=
    \frac{4qw_{soot}}{a f\kappa_{soot} \alpha_{out}^* }.
\end{align}
Tuhkakerros:
\begin{align}
    \frac{\doo \Delta P_{ash}}{\doo w_{soot}} &= 0,
    \\\frac{\doo \Delta P_{ash}}{\doo w_{ash}} &=
    \frac{2q}{a f \kappa_{ash}  }.
\end{align}

Näin ollen 
\begin{align}
    \frac{\doo \Delta P_{tot}}{\doo m_{soot}} &=
     \frac{1}{ac\rho_{soot}}
    \left(\frac{10 F L^2 d^2 q}{3 V \alpha_{in}^*} +
     \frac{2 q}{2 \kappa_{soot} \alpha_{out}^*}
    \right)\\
% \end{align}
% \begin{align}
\begin{split}
    \frac{\doo \Delta P_{tot}}{\doo m_{ash}} &=
    \frac{b-c}{abc\rho_{ash}}\left(\frac{10 F L^2 d^2 q}{3 V \alpha_{in}^*} +
     \frac{2 q}{2 \kappa_{soot} \alpha_{out}^*}
    \right) \\
    & + 
    \frac{1}{ab\rho_{ash}} \left(
        \frac{10 F L^2 d^2 q}{3 V \alpha_{in}^*} +
        \frac{4qw_{soot}}{a f \kappa_{soot} \alpha_{out}^*  } +
        \frac{2q}{a f \kappa_{ash}  }
    \right).
\end{split}
\end{align}
% Tuhkakerroksen paksuuden linearisointi saadaan muotoon
% \begin{align}
%     \tilde{w}_{ash}( m_{soot},  m_{ash}) =
%     \frac{1}{ \rho_{ash} \, ab}\Delta m_{ash}.
% \end{align}

% % \begin{align}
% %     w_{soot}(m_{soot}, m_{ash}) =
% %     \frac{b - c}{2}.
% % \end{align}

% \begin{align}
% \begin{split}
%     \tilde{w}_{soot}(m_{soot}, m_{ash}) = 
%     \frac{1}{  ac \rho_{{soot}} } \Delta m_{soot} +
%     \frac{b-c}{abc\rho_{{ash}} } \Delta m_{ash} 
% \end{split}
% \end{align}



