\chapter{Hiukkassuodattimen tilan estimointi}%
\label{ch:dpf_estimointi}


Tässä luvussa laaditaan EFK-estimointialgoritmi noki-, ja tuhkalatauksille. Algoritmin tilanyhtälöt ovat \eqref{eq:tilanyhtalo_noki} ja \(\dot{m}_{ash} =\dot{m}_{ash,in}\), ja mittausyhtälö on \eqref{eq:deltaP_total}. 
Esitellään EKF-algoritmin muuttujat
\begin{align*}
    x &= \bm{m_{soot} \\ m_{ash}},  
    \qquad  u = \bm{ \dot{m}_{soot, in} \\ \dot{m}_{ash,in} \\ c_{NO_2, in}} 
    \\
    z &= \Delta P_{meas},
    \qquad \hat{z} =  \Delta P_{tot} \\
\end{align*}
Lämpötila tunnetaan tarkasti, joten se voidaan jättää mallin parametriksi. Hapen konsentraatio on useita kertaluokkia korkeampi kuin typpidioksidin, joten sitä ei tarvita sisäänmenoksi. 
EKF-algoritmia voidaan käyttää, kunhan funktiot linearisoidaan. 
Painehäviömallin funktiot ovat derivoituvia kaikkialla, joten painehäviöyhtälöt voidaan linearisoida.
Linearisointi on toteutettu liitteessä \ref{ch:dP_linearisointi}.