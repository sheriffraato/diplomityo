\chapter{Hiukkassuodattimen tilan estimointi}%
\label{ch:dpf_estimointi}


Tässä luvussa laaditaan EFK-estimointialgoritmi noki-, ja tuhkalatauksille. Algoritmin tilanyhtälöt ovat \eqref{eq:tilanyhtalo_noki} ja \(\dot{m}_{ash} =\dot{m}_{ash,in}\), ja mittausyhtälö on \eqref{eq:deltaP_total}. 
Esitellään EKF-algoritmin muuttujat
\begin{align*}
    x &= \bm{m_{soot} \\ m_{ash}},  
    \qquad  u = \bm{m_{soot, in} \\ m_{ash,in} \\ c_{NO_2, in}} 
    \\
    z &= \Delta P_{meas},
    \qquad \hat{z} =  \Delta P_{tot} \\
\end{align*}
Lämpötila tunnetaan tarkasti, joten se voidaan jättää mallin parametriksi. Hapen konsentraatio on useita kertaluokkia korkeampi kuin typpidioksidin, joten sitä ei tarvita sisäänmenoksi. Tilanyhtälö on muotoa
\begin{align}
    \dot{x} = \bm{\dot{m}_{soot} \\ \dot{m}_{ash}} &=
    \bm{a &0 \\0 & 1} \bm{m_{soot} \\ m_{ash}} + 
    \bm{1 & 0 & b\\ 0 & 1 & 0} \bm{\dot{m}_{soot, in} \\ \dot{m}_{ash, in} \\ \dot{c}_{NO_2, in}},
\end{align}
jossa parametrit
\begin{align*}
    a &= -\frac{M_{soot}R_{active}(t)c_{O_2}(t)}{\rho_{soot}}, \qquad b = -\frac{M_{soot}R_{passive}(t)}{{\rho_{soot}}}.
\end{align*}
Tilanyhtälö on tilanmuuttujien suhteen lineaarinen, joten sitä voidaan käyttää sellaisenaan. Mittausyhtälö puolestaan ei ole. 
EKF-algoritmia voidaan käyttää, kunhan funktiot linearisoidaan. 
Painehäviömallin funktiot ovat derivoituvia kaikkialla, joten painehäviöyhtälöt voidaan linearisoida.
Linearisointi on toteutettu liitteessä \ref{ch:linearisointi}.